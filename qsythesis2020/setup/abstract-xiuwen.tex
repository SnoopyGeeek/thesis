% !Mode:: "TeX:UTF-8" 

%====================================== 中文摘要 ==========================================
\BiAppendixChapter{摘~~~~要}{ABSTRACT (Chinese)}
\setcounter{page}{1}\pagenumbering{Roman}
\defaultfont

%博士学位论文摘要正文为 1000 字左右。
%
%内容一般包括:从事这项研究工作的目的和意义;完成的工作 (作者独立进行的研究工作及相应结果的概括性叙述);获得的主要结论 (这是摘要的中心内容)。博士学位论文摘要应突出论文的创新点。
%
%摘要中一般不用图、表、化学结构式、非公知公用的符号和术语。
%
%如果论文的主体工作得到了有关基金资助,应在摘要第一页的页脚处标注:本研究得到某某基金 (编号:) 资助。

网络通信是构建当今社会的重要基础设施,当前的发展方向主要集中于建设高性能、高可创新性的网络架构。最近10年,软件定义网络(SDN)和可编程网络(SDN2.0)概念的提出很好地解决了过去网络创新难度大的问题,但随着流量和网络功能复杂度的快速提升,这种新的网络体系结构也带来了性能和鲁棒性两方面的挑战。性能方面:基于CPU 的转发平台性能发展逐步减慢,基于ASIC的智能网卡硬件可编程性差。鲁棒性方面:数据平面和控制平面分离的SDN网络架构带来了稳定性不足和安全性差、效率低的问题。

本文将问题从网络系统的三个维度进行分析:
(1)主机侧网络,在服务器网卡层面,基于CPU的智能网卡性能难以满足目前虚拟化技术和网络监管细粒度化的发展需求。
(2)交换侧网络,在核心网骨干网层面,基于ASIC的转发平面不足以提供网络处理的高灵活性。由于在成本、性能之间平衡困难,网络工程师的创新空间受到了限制。
(3)控制面与数据面交互,硬件流表是一种高效且昂贵的网络转发核心部件,在软件定义网络时代流表稀缺性更加突出。由于流数目和流量的快速增长,控制平面针对流表的操作导致大量控制通信开销。易导致网络鲁棒性差,易形成安全隐患。

近年来,现场可编程门阵列(FPGA)器件得到快速发展,以可编程硬件技术为首的异构架构已经大量融合到网络领域,带来高用户可定制能力的同时也能保证了一定的处理性能,这也为此论文的研究内容提供了基础的保障。
本文主要探索以可编程硬件为基础的高性能网络数据平面以及网络系统。本文研究在软件定义的网络编程框架内如何将这种可编程硬件抽象层融入整体系统,并设计与其配套的控制平面协议,使整体网络系统的软硬件有机结合,在增强网络处理能力和灵活性的同时保证安全和效率。本文由理论分析入手提出了体系架构,进而给出系统实现并进行验证。本文将从以下三方面阐述:

1.研究可编程设备加速主机侧网络方法。本文提出利用基于FPGA的智能网卡卸载操作系统内部分网络功能,达到扩展网络接入层的性能的目的。探讨了不同场景下网络功能的构成,分析并提出一种基于可编程硬件的流式计算模型(Data-Computing,DC抽象)。本文把服务器网络功能任务中可转化为DC抽象的计算密集型任务通过合理转换卸载到智能网卡。论文基于可编程网卡设计了一套网络流量捕获、统计分析和回放系统。在满足网络功能不改变的前提下,证明利用基于FPGA的智能网卡能有效地提升服务器的网络性能(100x)、抖动(降低$10^4$x)和效率(10x)。

2.研究可编程设备加速网络交换层方法。本文提出一种硬件异构型的可编程网络数据平面架构,将FPGA与ASIC交换芯片有机结合,增强ASIC处理报文的灵活性,同时满足高吞吐的性能需求。论文设计了ASIC面向可编程硬件的扩展接口。交换芯片将数据包头拆分并通过高速数据互联载体发送给FPGA,利用FPGA可重配特性实现完全可编程的包头处理;同时,本文基于DC抽象,将网络随路计算(network-centric computing)模式引入可编程网络体系架构;通过分析网络流量特征在FPGA中设计了一种并行化处理单元,在资源消耗可控的前提下大规模提高可编程硬件处理吞吐(120x)。

3.SDN网络硬件流表可扩展性研究。由可编程网卡和交换机组成的数据平面内,最重要的资源是流表资源。本文从SDN网络全局视野出发,着手解决流表资源匮乏的问题。本文分析不同的流量规模和特征,以及系统多模块之间的互联协议,提出一种转发设备节点之间的流表共享机制。实现了数据平面应对突发流量时的稳定性。本文将因流表资源不足引发的交换机转发RTT时间和安全通道消息风暴数量的优化均达到至少2个数量级。



{\boldsong}
\vspace{\baselineskip}
\noindent{\fontsize{11.5pt}{11.5pt}\boldsong 关\hspace{0.5em}键\hspace{0.5em}词}:软件定义网络;网络数据平面;可编程硬件;现场可编程门阵列;流表

\vspace{\baselineskip}
\noindent{\fontsize{11.5pt}{11.5pt}\boldsong 论文类型}:应用基础

%论文类型包括:a.理论研究(Theoretical Research);b.应用基础(Application Fundamentals);c.应用研究(Application Research);d.研究报告(Research Report);e.设计报告(Design Report);f.案例分析(Case Study);g.调研报告(Investigation Report);h.产品研发(Product Development);i.工程设计(Engineering Design);j.工程/项目管理(Engineering/Project Management);k.其它(Others)。

\clearpage

%====================================== 英文摘要 ==========================================
\BiAppendixChapter{ABSTRACT}{ABSTRACT (English)}

\noindent 英文摘要正文每段开头不缩进,每段之间空一行。\newline

\noindent The abstract goes here. \newline

\noindent \LaTeX{} is a typesetting system that is very suitable for producing scientific and mathematical documents of high typographical quality.

%\noindent You will never want to use Word when you have learned how to use \LaTeX.

\vspace{\baselineskip}
\noindent{\fontsize{11.5pt}{11.5pt}\selectfont\bfseries KEY WORDS}: Xi'an Jiaotong University, Doctoral dissertation, \LaTeX{} template

\vspace{\baselineskip}
\noindent{\fontsize{11.5pt}{11.5pt}\selectfont\bfseries TYPE OF DISSERTATION}: Application Fundamentals

\clearpage
