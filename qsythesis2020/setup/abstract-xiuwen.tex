% !Mode:: "TeX:UTF-8" 

%====================================== 中文摘要 ==========================================
\BiAppendixChapter{摘~~~~要}{ABSTRACT (Chinese)}
\setcounter{page}{1}\pagenumbering{Roman}
\defaultfont

%博士学位论文摘要正文为 1000 字左右。
%
%内容一般包括:从事这项研究工作的目的和意义;完成的工作 (作者独立进行的研究工作及相应结果的概括性叙述);获得的主要结论 (这是摘要的中心内容)。博士学位论文摘要应突出论文的创新点。
%
%摘要中一般不用图、表、化学结构式、非公知公用的符号和术语。
%
%如果论文的主体工作得到了有关基金资助,应在摘要第一页的页脚处标注:本研究得到某某基金 (编号:) 资助。

软件定义网络概念的提出,推动网络向高性能和高可编程的方向发展。
软件定义网络利用流表建立起“匹配+执行”的编程抽象方法,解耦网络数据平面与控制平面;
数据包协议无关处理器通过扩展匹配域定义的方式,进一步优化了网络可编程抽象能力。
可编程网络架构的控制平面与数据平面分离,数据平面转发设备中的流表是表达控制算法灵活性的关键部件。但高性能硬件流表容量较低,易导致节点流表溢出从而影响转发性能;
数据包协议无关的处理技术扩展了数据包头的匹配灵活度,但无法满足数据平面内其他可编程计算需求。因此目前网络数据平面设备面临着吞吐性能和灵活性两方面的制约。

本文主要研究在网络数据平面内如何利用可编程硬件来更灵活、更高效地支持多种类别的可编程抽象方法。
本文首先以解决流表资源瓶颈问题为基础,接着利用可编程硬件分别在网络中间节点、主机侧网络中构建高性能与高可编程性协同发展的方法,最终形成一套完整的端到端网络系统。
具体贡献如下:

%{摘要中的贡献、绪论中的创新点,以及总结展望中的总结相互对应,换种表述方式而已。最好用一种套路写法,例如:“为应对XXX问题,提出一种XXX方法,该方法的基本思路,方法的性能或效果。”}

1.提出一种全局场景下的流表资源可扩展方法(Flow Table Sharing, FTS)。
为解决单个转发节点流表容量受限以及流表溢出后数据包处理性能严重下降的问题,本文提出流表共享机制FTS。
FTS通过与邻居节点建立流表共享,提升整体流表资源的动态利用率,同时为由于没有足够空间而无法建立流表项的流量,设置一种基于离线转发策略的组表转发方式。
在流表溢出的情况下,该方案与传统OpenFlow协议相比,使OpenFlow交换机转发RTT延迟和安全通道控制报文风暴数量的优化均达到至少2个数量级。

2.提出一种支持自适应计算的硬件交换系统架构(Adaptable Switch, AS)。
针对高性能网络转发设备可编程灵活性差的问题,本文提出利用可编程硬件加速网络中间节点的普适性计算的方法AS。
利用高性能的转发芯片与可编程硬件有机结合形成异构体,同时获得性能提升以及更高的可编程灵活性。在此基础上本文还提出了一套部署在硬件上的高资源利用率的并行流水线和流表分配优化算法。自适应交换系统在满足全可编程灵活性的条件下,与基于FPGA的数据平面相比,将数据包处理性能提升120倍以上。

3.提出一种端侧网络在网络测量领域的可编程抽象方法。
网络测量是众多网络功能的基础,针对目前测量工具性能低下以及缺乏统一的编程抽象,为了提高系统使用灵活性以及可用性,本文针对大量场景提出了一套基于测量技术的网卡硬件流水线系统。通过不同的软件调用,令系统硬件适用于高性能的网络安全、访问控制、流量控制、拥塞探测等多种场景。
考虑到高性能硬件资源有限的问题,
论文将可编程测量系统抽象为基础的包个数与数据量的压缩统计并提供不同的统计触发方法。
在节约38\%的硬件存储空间情况下,相较于软件的处理方式系统吞吐率提升8倍、系统的处理能耗节约90\%。

综上所述,本文针对可编程网络不同侧面利用了可编程硬件技术,对可编程网络内核心运算资源不足的问题、核心交换节点可编程能力弱的问题以及可编程网络编程复杂度高的问题进行了研究,从网络系统的管理层协议、网络数据平面异构架构以及可编程网络的编程抽象三方面提升了可编程网络系统在真实网络场景下的可用性。



{\boldsong}
\vspace{\baselineskip}
\noindent{\fontsize{11.5pt}{11.5pt}\boldsong 关\hspace{0.5em}键\hspace{0.5em}词}:软件定义网络;网络数据平面;可编程硬件;现场可编程门阵列;流表

\vspace{\baselineskip}
\noindent{\fontsize{11.5pt}{11.5pt}\boldsong 论文类型}:应用基础

%论文类型包括:a.理论研究(Theoretical Research);b.应用基础(Application Fundamentals);c.应用研究(Application Research);d.研究报告(Research Report);e.设计报告(Design Report);f.案例分析(Case Study);g.调研报告(Investigation Report);h.产品研发(Product Development);i.工程设计(Engineering Design);j.工程/项目管理(Engineering/Project Management);k.其它(Others)。

\clearpage

%====================================== 英文摘要 ==========================================
\BiAppendixChapter{ABSTRACT}{ABSTRACT (English)}

%\noindent 英文摘要正文每段开头不缩进,每段之间空一行。\newline
%
%\noindent The abstract goes here. \newline
\noindent The concept of software-defined networking has promoted the development of high-performance and high-programmability networks.
The software-defined network uses flow tables to establish an abstract programming method of "matching + execution" to decouple the network data plane and control plane;
The packet protocol-independent processor further optimizes the network programmable abstraction ability by extending the matching domain definition.
The control plane of the programmable network architecture is separated from the data plane, and the flow table in the data plane forwarding device is a key component that expresses the flexibility of the control algorithm. However, the high-performance hardware flow table has a low capacity, which can easily cause the node flow table to overflow and affect the forwarding performance;
The data packet protocol-independent processing technology expands the matching flexibility of the data packet header, but cannot meet other programmable computing requirements in the data plane. Therefore, the current network data plane equipment is facing the constraints of throughput performance and flexibility.\newline

\noindent This article mainly studies how to use programmable hardware in the network data plane to support multiple types of programmable abstraction methods more flexibly and efficiently.
This article is based on solving the bottleneck problem of flow table resources, and uses programmable hardware to build a coordinated development of high performance and high programmability methods in the network intermediate nodes and the host side network, and finally form a complete end-to-end network system.
The specific contributions are as follows:\newline


\noindent 1. Propose an extensible method for flow table resources (Flow Table Sharing, FTS) in a global scenario.
In order to solve the problem that the capacity of the flow table of a single forwarding node is limited and the performance of data packet processing after the flow table overflows, the flow table sharing mechanism FTS is proposed in this paper.
FTS establishes flow table sharing with neighbor nodes to improve the dynamic utilization of the overall flow table resources. At the same time, it sets a group table forwarding method based on an offline forwarding strategy for the traffic that cannot establish flow entries due to insufficient space.
In the case of flow table overflow, compared with the traditional OpenFlow protocol, this solution enables the optimization of the OpenFlow switch forwarding RTT delay and the number of secure channel control packet storms to reach at least 2 orders of magnitude.\newline

\noindent 2. Supporting adaptive computing hardware switching system architecture (Adaptable Switch, AS).
Aiming at the problem of poor programmable flexibility of high-performance network forwarding equipment, this paper proposes a method AS, which uses programmable hardware to accelerate the universal computing of network intermediate nodes.
Use high-performance forwarding chips and programmable hardware to organically combine to form isomers, while gaining performance improvements and higher programmable flexibility. It also proposes a set of parallel pipelines and flow table allocation optimization algorithms with high resource utilization deployed on hardware, so that the adaptive switching system meets the conditions of full programmable flexibility, compared with FPGA-based data planes. Improve data packet processing performance by more than 120 times.\newline

\noindent 3. The programmable abstract method of end-side network in the field of network measurement.
Network measurement is the basis of many network functions. In view of the low performance of current measurement tools and the lack of unified programming abstraction, in order to improve the flexibility and usability of the system, this paper proposes a network card hardware pipeline system based on measurement technology for a large number of scenarios. Through different software calls, the system hardware is suitable for high-performance network security, access control, flow control, congestion detection and other scenarios.
Considering the limited resources of high-performance hardware,
The thesis abstracts the programmable measurement system based on the compression statistics of the number of packets and data volume and provides different statistical trigger methods.\newline
In the case of saving 38\% of the hardware storage space, the system throughput rate is increased by 8 times compared with the software processing method, and the processing energy consumption of the system is reduced by 90\%.

%\noindent You will never want to use Word when you have learned how to use \LaTeX.

\vspace{\baselineskip}
\noindent{\fontsize{11.5pt}{11.5pt}\selectfont\bfseries KEY WORDS}: Software Defined network, Data Plane, Hardware-based programmability, FPGA, Flow Table



\vspace{\baselineskip}
\noindent{\fontsize{11.5pt}{11.5pt}\selectfont\bfseries TYPE OF DISSERTATION}: Application Fundamentals

\clearpage
