% !Mode:: "TeX:UTF-8" 

\BiAppendixChapter{致\quad 谢}{Acknowledgements}


写到这里意味着已经完成了研究生阶段全部工作,回顾读博士7年的时间,我于专业领域的知识与技术都得到了不小的进步。几千天在人生中不长,但也不短,一夜夜的点灯熬油、一行行的实验代码以及一篇篇的文章专著都在我人生中留下了难忘的记忆。


首先感谢我一不留神就呆了10年的母校西安交通大学,她严谨的办学理念、优良的学风传统以及十六字校训为我打下为人为学的根基,也给了我深入地思考问题的空间。

导师是学海中的灯塔,感谢邹建华导师对我悉心照顾与学业上的帮助。感谢管晓宏院士为我的研究生涯指明方向,特别感谢您对我博士论文提出认真的修改意见,您对学术论文表达高屋建瓴令我启发深刻,您的每一句教导使我受益匪浅。感谢胡成臣教授对我科研和学习上的指导,他敏锐的思维和富有洞察力的视野给我深刻启迪。

在课题组的这段时间,交到了一辈子的朋友,一同学习也一直向大家学习:张帆、孙秀文、许琛、郑鹏、王瑞龙。

还要感谢在新加坡赛灵思亚太研究院的导师和同事们:Gordon、Yan、Henry、Nguyen,近两年的海外访问实习经历令我的眼界及职业素养有长足进步。

感谢一直关心我的父母,你们的奉献、支持、鼓励和陪伴给了我莫大动力。

博士不只是一个头衔,更要有一种自主科研、分析和学习的能力。以开放的心态乐于接收新的知识碰撞思想的火花。善于发现自己的不足并富有探索精神,才是符合新时代青年的特征,愿为祖国繁荣昌盛而奋斗。

人生难得一位伴侣和知己,瑾以本文献给叶乃馨女士。

%\vspace{1em}
%{\color{red} 用于盲审的论文,此页内容全部隐去}。
