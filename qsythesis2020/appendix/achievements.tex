% !Mode:: "TeX:UTF-8" 

\defaultfont
\BiAppendixChapter{攻读学位期间取得的研究成果}{Achievements}




\noindent 论文:
\begin{publist}
	\item \textbf{Qiao Siyi}, Hu Chengchen, Gordon Brebner, Zou Jianhua, Guan Xiaohong. Adaptable Switch: A Heterogeneous Switch Architecture for Network-Centric Computing[J].IEEE Communications Magazine. 2020,58(12):64-69.
	
	(IF=11.052, SCI:, DOI:https://doi.org/10.1109/MCOM.001.2000399).
	\item {\hei 乔思祎},胡成臣,李昊,管晓宏,邹建华.OpenFlow交换机流表溢出问题的缓解机制[J]. 计算机学报,2018,41(9):2003-2015.
	
	(CCF A, EI: 20184406006908, DOI: 10.11897/SP.J.1016.2018.02003).
	%
	\item \textbf{Qiao Siyi}, Hu Chengchen, Guan Xiaohong, Zou Jianhua. Taming the flow table overflow in openflow switch[C].In Proceedings of the 2016 ACM SIGCOMM Conference. 2016: 591-592.\\
	(CCF A, EI: 20163702808112, DOI: https://doi.org/10.1145/2934872.2959063).
	%
	\item \textbf{Qiao Siyi}, Xu Chen, Xie Lei, Yang Ji, Hu Chengchen, Guan Xiaohong, Zou Jianhua. Network recorder and player: FPGA-based network traffic capture and replay[C].2014 International Conference on Field-Programmable Technology (FPT). IEEE, 2014: 342-345.
	
	(CCF C, EI: 20151900816423, DOI: 10.1109/ FPT.2014.7082815).
\end{publist}

\noindent 科研获奖:
\begin{publist}
	\item {\hei 乔思祎}(首位).OpenFlow交换机流表溢出问题的缓解机制.第五届中国互联网学术年会最佳学生论文奖. 中国计算机学会 互联网专业委员会.2016-08-15.
\end{publist}

\noindent 专利:
\begin{publist}
	
	\item 胡成臣,孙秀文,{\hei 乔思祎},李昊. 一种基于FPGA平台的压缩流量模式匹配引擎及模式匹配方法[P]. 陕西省:CN110865970A, 2020-03-06.
	\item Hu Chengchen, Yang Ji, Zhang Yan, \textbf{Qiao Siyi}. A hardware architecture for passive network measurement based on flow statistics. US patent, 2019 (已申请).
	\item 胡成臣,杨骥,龚志敏,杨卫,赵泓博,{\hei 乔思祎},张丽山,徐友庆,吕伟男. 一种全可编程SDN高速网卡[P]. 江苏:CN204392269U, 2015-06-10. 
	%
	\item 胡成臣,赵泓博,吕伟男,杨骥,团哲恒,史明,{\hei 乔思祎},杨卫. 一种全可编程SDN交换机[P]. 江苏:CN204168323U, 2015-02-18.
\end{publist}






\vspace{\baselineskip}
%{\color{red} 用于盲审的论文,只列出已发表学术论文的题目和刊物名称,可以备注自己为第几作者,及期刊影响因子。}

\clearpage{\pagestyle{empty}\cleardoublepage}% 声明从奇数页开始
